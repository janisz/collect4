% This is based on the LLNCS.DEM the demonstration file of
% the LaTeX macro package from Springer-Verlag
% for Lecture Notes in Computer Science,
% version 2.4 for LaTeX2e as of 16. April 2010
%
% See http://www.springer.com/computer/lncs/lncs+authors?SGWID=0-40209-0-0-0
% for the full guidelines.
%

\documentclass{llncs}

\usepackage[utf8]{inputenc}
\usepackage[T1]{fontenc}
\usepackage[polish]{babel}
\usepackage[colorinlistoftodos]{todonotes}
\usepackage{hyperref}
\usepackage{subfig}

\begin{document}

\title{Ewolucyjna konstrukcja modeli neuronowych}
%
\titlerunning{ConvConnect4}  % abbreviated title (for running head)
%                                     also used for the TOC unless
%                                     \toctitle is used
%
\author{Tomasz Janiszewski\inst{1} \and Jakub Dutkowski\inst{2}}
%
\authorrunning{Tomasz Janiszewski, Jakub Dutkowski} % abbreviated author list (for running head)
%
%%%% list of authors for the TOC (use if author list has to be modified)
\tocauthor{Tomasz Janiszewski and Jakub Dutkowski}

\institute{Politechnika Warszawska\\
Wydział Matematyki i Nauk Informacyjnych
\email{janiszewskit@student.mini.pw.edu.pl}~~
\email{dutkowskij@student.mini.pw.edu.pl}}

\maketitle              % typeset the title of the contribution

\keywords{Connect4, Neural Network, Evolutionary Algorithm, Neuroevolution, Connect Four}
%
\section{Zadanie}
Zbadać efektywność zastosowania programowania genetycznego do konstrukcji neuronowych sieci MLP 
(perceptronów wielowarstwowych) z pojedynczą warstwą ukrytą.
Eksperymenty oprzeć na wybranym zestawie zadań regresji.

\section{Wstęp}
Celem projektu jest zbadanie przydatności programowania genetycznego do konstrukcji neuronowych sieci MLP.
W projekcie skupimy się na budowaniu sieci grającej w Connect Four (\emph{pol. Czwórki}) \cite{connect4:wiki}
na planszy o rozmiarze $4 \times 4$.

\section{Opis problemu}
Podczas konstrukcji sieci neuronowych istnieje problem doboru parametrów tej sieci takich jak:
\begin{itemize}
	\item ilość warstw ukrytych
	\item liczba neuronów w każdej warstwie
	\item funkcja aktywacji neuronu
	\item obecność biasu
	\item inicjalne wagi
	\item współczynniki nauki
\end{itemize}
aby jak proces uczenia był jak najbardziej efektywny. Algorytmy opartę na metodzie propagacji wstecznej osiągają
dość dobre rezultaty, jednak nie są odporne na utykanie w ekstremach lokalnych oraz osiągają małą zbieżność
na płaskich odcinkach funkcji celu. Aby wyeliminować te problemy stosuje się takie techniki jak symulowane wyżażanie.
W nimniejszej pracy chcemy się zająć użyciem algorytmu ewolucyjnego aby znaleźć optymalną strukturę sieci
dla danego problemu przeszukując jak najmniejszy obszar.

\section{Sieć neuronowa}
Badana rodzina sieci neuronowych będzie grała w Connect Four na planszy o rozmiarze $4 \times 4$. Plansza będzie zakodowana
jako tablica szesnastu ($16$) elementów, z których każdy może przyjmować wartości $-1,0,1$ oznaczające
\begin{itemize}
	\item[$-1$] pole zajęte przez przeciwnika
	\item[$0$] pole wolne
	\item[$1$] pole zajęte przez gracza
\end{itemize}
Na wyjściu sieci dla zadanej planszy znajduje się liczba oznaczająca ruch dla danego na wejściu stanu planszy.

\section{Algorytm uczenia}
Sieć będzie uczona za pomocą algorytmu propagacji wstecznej. Dane treningowe wygenerowane zostaną za pomocą programu programu
opartego o algorytm alfa-beta.

\section{Algorytm ewolucyjny}
W projekcie zastosujemy klasyczny algorytm genetyczny.
\subsection{Fenotyp}
\begin{itemize}
	\item funkcja aktywacji
	\item ilość neuronów w warstwie ukrytej
	\item wektor wag
\end{itemize}

\subsection{Genotyp}
\begin{itemize}
	\item funkcja aktywacji -- 3 elementowy wektor binarny okręslający funkcje wyjścia poszczególnych warstw
	\item ilość neuronów w warstwie ukrytej -- dodatnia liczba całkowita
	\item wektor wag -- dodatnia liczba całkowita określająca ziarno losowania wektora wag
	\item bias -- 3 elementowy wektor binarny okręslający obecność biasu dla poszczególnych warstw
\end{itemize}

\begin{example}

\begin{tabular}{ | c | c | c | c | c | c | c | c | }
	\hline
	0 & 1 & 0 & 5 & 2356 & 0 & 1 & 1\\
	\hline
\end{tabular}
\\
Powyższa definicja oznacza sieć neuronową złożoną z 3 warstw 
(wejście -- 4 neurony, warstwa ukryta -- 5, wyjście -- 1).
Warstwa wejściowy i wyjściowa kożystają z tangensa hiperbolicznego jako funkcji przejścia natomiast 
warstwa ukryta używa funkcji sigmoidalnej. Bias jest obecny w drugiej i trzeciej warstwie. A ziarno
generatora liczb losowych ustawione jest na $2356$.
\end{example}

\subsection{Krzyżowanie}
Krzyżowanie funkcji aktywacji będzie i biasu oparte będzie na krzyżowaniu dwupunktowym. Natomiast
krzyżowanie ilości neuronów w warstwie ukrytej i wektora wag na krzyżowaniu uśredniającym.

\subsection{Mutacje}
Mutacja funkcji przejśćia i biasu będzie polegała na zamianie bitu na losowy.
Mutacja ilości neuronów w warstwie ukrytej i wektora wag będzie polegało na zmianie
wartości o losowy współczynnik.

\subsection{Funkcja przystosowania}
Wartością funkcji przystosowania będzie błąd na zbiorze testowym po ustalonym treningu.

\subsection{Metoda selekcji}
Do selekcji zostanie użyta metoda ruletki.

%
% ---- Bibliography ----
%
\begin{thebibliography}{4}
%
\bibitem {arabas}
Arabas J.:
\textsl{Wykłady z algorytmów ewolucyjnych}, WNT Warszawa,2001
\bibitem {mandziuk2011}
Mandziuk, J.:
\textsl{Towards Cognitively Plausible Game Playing Systems}
IEEE COMPUTATIONAL INTELLIGENCE MAGAZINE, Maj, 2011
\bibitem {connect4:wiki}
\textsl{Connect Four --- {W}ikipedia{,} The Free Encyclopedia}
\url{http://en.wikipedia.org/wiki/Connect_Four}
\bibitem {velena}
\textsl{A Shannon C-type program which plays connect four perfectly}
\url{http://www.ce.unipr.it/~gbe/velena.html}
\end{thebibliography}

\end{document}
